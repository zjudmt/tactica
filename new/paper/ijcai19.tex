%%%% ijcai19.tex

\typeout{IJCAI-19 Instructions for Authors}

% These are the instructions for authors for IJCAI-19.

\documentclass{article}
\pdfpagewidth=8.5in
\pdfpageheight=11in
% The file ijcai19.sty is NOT the same than previous years'
\usepackage{ijcai19}

% Use the postscript times font!
\usepackage{times}
\usepackage{soul}
\usepackage{url}
\usepackage[hidelinks]{hyperref}
\usepackage[utf8]{inputenc}
\usepackage[small]{caption}
\usepackage{graphicx}
\usepackage{amsmath}
\usepackage{booktabs}
\usepackage{algorithm}
\usepackage{algorithmic}
\usepackage{subcaption}
\urlstyle{same}

% the following package is optional:
%\usepackage{latexsym} 

% Following comment is from ijcai97-submit.tex:
% The preparation of these files was supported by Schlumberger Palo Alto
% Research, AT\&T Bell Laboratories, and Morgan Kaufmann Publishers.
% Shirley Jowell, of Morgan Kaufmann Publishers, and Peter F.
% Patel-Schneider, of AT\&T Bell Laboratories collaborated on their
% preparation.

% These instructions can be modified and used in other conferences as long
% as credit to the authors and supporting agencies is retained, this notice
% is not changed, and further modification or reuse is not restricted.
% Neither Shirley Jowell nor Peter F. Patel-Schneider can be listed as
% contacts for providing assistance without their prior permission.

% To use for other conferences, change references to files and the
% conference appropriate and use other authors, contacts, publishers, and
% organizations.
% Also change the deadline and address for returning papers and the length and
% page charge instructions.
% Put where the files are available in the appropriate places.

\title{Formation2Vec: Exploring a Representation for Formation Segmentation and Detection in Soccer Games}

% Single author syntax
\author{
    Sarit Kraus
    \affiliations
    Department of Computer Science, Bar-Ilan University, Israel \emails
    pcchair@ijcai19.org
}

% Multiple author syntax (remove the single-author syntax above and the \iffalse ... \fi here)
% Check the ijcai19-multiauthor.tex file for detailed instructions
\iffalse
\author{
First Author$^1$
\and
Second Author$^2$\and
Third Author$^{2,3}$\And
Fourth Author$^4$
\affiliations
$^1$First Affiliation\\
$^2$Second Affiliation\\
$^3$Third Affiliation\\
$^4$Fourth Affiliation
\emails
\{first, second\}@example.com,
third@other.example.com,
fourth@example.com
}
\fi

\begin{document}

\maketitle

\begin{abstract}
    In the soccer game, formation depicts the role of the soccer players on the pitch, revealing the basic tactic of the teams. During a game, a team will switch between offensive and defensive formations as they get or lose the ball. Although the analysis of formation changing is important in soccer, there are few methods to detect and classify the formations automatically. In this paper, we analogize the formation detection to the human action segmentation and classification problem. Referring to the general framework to solve such a problem, we firstly absorb a neighborhood awareness method to model the positional data and obtain the topological structure of the players, which can help represent the position. Our experiments prove that the representation outperforms the alternative representations, e.g. player positions and parameterized pitch. Then by a convolutional network, the temporal patterns will be captured, enabling us to segment the game into the periods with specific formations. The accuracy of the method achieves a satisfactory level in action segmentation and classification area. Compared with the existing methods in formation detection, our method supports fine-grained analysis of formation change and shows a wider application prospect.

\end{abstract}

\section{Introduction}

Soccer is the most popular sports in the world with huge commercial value. This 11-a-side competition shows a high degree of confrontation and dynamic. The players on the field (except goalkeeper) are organized by the term {\bf formation}, which is arranged by the coach considering the characteristics of the players and the situation of the game. The formation of a team is described using the form $n_b$-$n_m$-$n_f$, which means that $n_b$ backwards, $n_m$ midfielders and $n_f$ forwards (or $n_b$-$n_{dm}$-$n_{om}$-$n_{f}$, which means $n_b$ backwards, $n_{dm}$ defensive midfielders, $n_{om}$ offensive midfielders and $n_f$ forwards). For example, Barcelona from Spanish La Liga usually uses 4-3-3 formation this season and Liverpool from English Premier League prefers 4-2-3-1. Actually during a game, the formation of a team is not fixed. As the team obtains the ball, the players will move forward and the number of people in the frontcourt will increase; when the opponents attack and approach the goal, the team will adopt a deep defensive formation, like 6-3-1, to enhance the defense. 

In this paper, we try to solve the problem of formation detection with positional data of the players. To be more specifically, we explored a new method to automatically segment the game into periods, each with a unique label indicating the formation of the team (including no formation). This problem is somehow similar to human action segmentation and detection, which generally extract the features from the video frame by frame and then learn the temporal patterns by training classifiers\cite{lea2017temporal}. Different from the human action segmentation and detection problem, the formtaiton detection poses several challenges. First, the representation of positional data that best fit for formation detection is not clear. The original data format is the $x$, $y$ coordinates of the players projected to a 1050 $\times$ 680 pixel 2D soccer field. To improve the accuracy of the classification, a more high-level feature representation that captures the topological characteristics of the positional data is needed.
Second, the context information is crucial to detect a formation. At the time of formation transformation, the distribution of the players in a single frame may look like some other formation. So the model is supposed to capture the context feature of the data and segment the periods of time with stable formation. 
Third, the data labelling needs to be done by professional soccer analysts. Unlike human action, it's difficult to determine when a formation is formed even manually, because a team usually takes 5 to 8 seconds to transform from one formation to another formation when they get or lose the ball.

To address the aforementioned challenges, we propose a new representation of the positional data, which can capture the topological feature of the unstructured data. We then use temporal convolutional networks to learn the temporal patterns of the features.

\section{Related Works}
With the great development of computer vision and portable sensor, the experts in soccer are able to get access to the positional data of the players and dig deep into the tactics of the soccer game. Most of the works focused on the passing pattern of the teams. Given the passing events and positional data of the players, \cite{Wang:2015} developed a specific topic model and found 10 main passing patterns of the players of Barcelona. With a same data type, \cite{Decroos:2018} defined the phases of events and used hierarchical agglomerative clustering to extract the attcking patterns. To facilitate the analysis, \cite{wei2013large} firstly developed a decision forest to segment the game into inplays and stoppages based on the player positions, team centroid and ball positions. Then they used formation as a prior knowledge to explore the offensive and conceding patterns of a team. 

Few works focus on formation analysis. \cite{wu2019forvizor} developed a visual analytic system with a tailored sankey diagram to visualize the formation change. The formations are extracted by a two-step clustering developed by \cite{bialkowski2014large}. The extraction consists of a k-means-like clustering and a hierarchical agglomerative clustering. However, they considered the formation as a stable factor, at least for a half game, which is not suitable for a fine-grained analysis of formation.

TCN (Temporal Convolutional Networks) are new methods designed for sequential data segmentation and classification. \cite{lea2017temporal} proposed TCN and achieved better results than previous state of the art in the field of human action detection and segmentation .\cite{gehring2017convolutional} from Facebook and \cite{kalchbrenner2016neural} from Google applied similar strategy to machine translation. \cite{bai2018empirical} evaluated TCNs in wider range of problems and disclosed the potential of TCNs, but the application of TCNs regarding formation detections was not discussed.

\section{Data Preprocessing and Description}
Our data is collected from the panoramic videos of soccer games on Pixellot\footnote{https://www.pixellot.tv/}. We used a semi-automatical collecting method to label the position of the players in the videos. To make sure the efficiency and preserve the precision of the labelling, a particle filter based tracking algorithm,developed by \cite{dearden2006tracking}, is ran under the supervision of the annotators. When occlusion happens and the tracker loses the target, the annotators have to mannually stop and correct the position of the tracker and make sure the tracking is accurate.

With the position of the players in the panoramic videos, we obtain a affine projection to map the position data to a two dimensional soccer pitch. With the position of the players in the panoramic videos, we used an affine projection to map the position data to a two dimensional soccer pitch. 
According to the latest \emph{Law of the Game 2018/19}\footnote{https://img.fifa.com/image/upload/khhloe2xoigyna8juxw3.pdf} by the International Football Association Board (IFAB), the length of the pitch (touchline) is between 100m and 110m and the width (goal line) is between 64 and 75m. But in many important international competitions, such as 2018 Russia World Cup, the pitch dimension is 105m by 68m. So we normalized the positional data into a 105m $\times$ 68m area.

The events on the pitch, such as foul, out of line and goal, are also very important for game analysis. The offensive or defensive state of a team may be interrupted by the events because the game will stop and kick off. With the events, we segment the game into small periods, and the time between the periods is stoppage time. Each period is independent to other periods. The position change within a period can be viewed as a sequential process.

\section{Approch}
Formation represents the role of the players on the pitch. A natural idea is regarding the players as the nodes in a network and convert the problem to a graph embedding problem. However, the definition of formation doesn't illustrate the connection between the players, which means that there is no edges between the nodes (players). According to the experience of the domain expert, the players in the same line, such as backward line, don't necessarily have connections during a game, such as passing. In practice, when a domain expert needs to determine the formation of a team, he will take a look at the evolution of the relative position of the players within a period. So the formaition detection is a context aware problem and the model have to take the previous positions of the players into consideration. Our task becomes finding a suitable representation of the relative position of the players and a model to learn the temporal pattern of the representations.

\subsection{Topological Representation}
\begin{figure}
    \centering
    \begin{subfigure}[b]{0.2\textwidth}
        \includegraphics[width=\textwidth]{pictures/8-c.png}
        \caption{Moore Neighborhood}
        \label{fig:moore1}
    \end{subfigure}
    \begin{subfigure}[b]{0.2\textwidth}
        \includegraphics[width=\textwidth]{pictures/8-c.png}
        \caption{Player Neighborhood}
        \label{fig:moore2}
    \end{subfigure}
    \caption{An illustration of Moore neighborhood}\label{fig:moores}
\end{figure}
To capture the topological features of the player positions, we borrow the idea of moore neighborhood in cellular automata theory. 
The original definition of the moore neighborhood is the cellular itself and the surrounding eight cells, as Figure~\ref{fig:moore1}. The mathematical definition of the Moore neighborhood with a radius $k$ is
\begin{equation}
    \{c_{ij}| \|c_{ij}-c\|_{L_{\infty}}<k\},
\end{equation}
% \{c_{ij}| \|c_{ij}-c\|_{L_{\infty}}<k\},
where $c_{ij}$ is the cell on the $i$-th row and the $j$-th column ,$c$ is the center of the Moore neighborhood and $L_{\infty}$ is the Chebeshev distance.
% Considering a player on the pitch as a cellular, w

% In soccer games, the size of the pitch is The 2D positional data of the players in our paper is normalized into a 1050 $\times$
We firstly parameterize the soccer pitch.




% Print manuscripts two columns to a page, in the manner in which these
% instructions are printed. The exact dimensions for pages are:
% \begin{itemize}
% \item left and right margins: .75$''$
% \item column width: 3.375$''$
% \item gap between columns: .25$''$
% \item top margin---first page: 1.375$''$
% \item top margin---other pages: .75$''$
% \item bottom margin: 1.25$''$
% \item column height---first page: 6.625$''$
% \item column height---other pages: 9$''$
% \end{itemize}


% \begin{quote} 
% \mbox{\tt $\backslash$usepackage\{times\}}
% \end{quote}
% in the preamble.\footnote{You may want also to use the package {\tt
% latexsym}, which defines all symbols known from the old \LaTeX{}
% version.}
% Additionally, it is of utmost importance to specify the American {\bf
% letter} format (corresponding to 8-1/2$''$ $\times$ 11$''$) when
% formatting the paper. When working with {\tt dvips}, for instance, one
% should specify {\tt -t letter}.

% \subsection{Title and Author Information}

% Center the title on the entire width of the page in a 14-point bold
% font. The title should be capitalized using Title Case. Below it, center author name(s) in  12-point bold font. On the following line(s) place the affiliations, each affiliation on its own line using 12-point regular font. Matching between authors and affiliations can be done using numeric superindices. Optionally, a comma-separated list of email addresses follows the affiliation(s) line(s), using  12-point regular font.

% \subsubsection{Blind Review}

% In order to make blind reviewing possible, authors must omit their
% names and affiliations when submitting the paper for review. In place
% of names and affiliations, provide a list of content areas. When
% referring to one's own work, use the third person rather than the
% first person. For example, say, ``Previously,
% Gottlob~\shortcite{gottlob:nonmon} has shown that\ldots'', rather
% than, ``In our previous work~\cite{gottlob:nonmon}, we have shown
% that\ldots'' Try to avoid including any information in the body of the
% paper or references that would identify the authors or their
% institutions. Such information can be added to the final camera-ready
% version for publication.

% \subsection{Abstract}

% Place the abstract at the beginning of the first column 3$''$ from the
% top of the page, unless that does not leave enough room for the title
% and author information. Use a slightly smaller width than in the body
% of the paper. Head the abstract with ``Abstract'' centered above the
% body of the abstract in a 12-point bold font. The body of the abstract
% should be in the same font as the body of the paper.

% The abstract should be a concise, one-paragraph summary describing the
% general thesis and conclusion of your paper. A reader should be able
% to learn the purpose of the paper and the reason for its importance
% from the abstract. The abstract should be no more than 200 words long.

% \subsection{Text}

% The main body of the text immediately follows the abstract. Use
% 10-point type in a clear, readable font with 1-point leading (10 on
% 11).

% Indent when starting a new paragraph, except after major headings.

% \subsection{Headings and Sections}

% When necessary, headings should be used to separate major sections of
% your paper. (These instructions use many headings to demonstrate their
% appearance; your paper should have fewer headings.). All headings should be capitalized using Title Case.

% \subsubsection{Section Headings}

% Print section headings in 12-point bold type in the style shown in
% these instructions. Leave a blank space of approximately 10 points
% above and 4 points below section headings.  Number sections with
% arabic numerals.

% \subsubsection{Subsection Headings}

% Print subsection headings in 11-point bold type. Leave a blank space
% of approximately 8 points above and 3 points below subsection
% headings. Number subsections with the section number and the
% subsection number (in arabic numerals) separated by a
% period.

% \subsubsection{Subsubsection Headings}

% Print subsubsection headings in 10-point bold type. Leave a blank
% space of approximately 6 points above subsubsection headings. Do not
% number subsubsections.

% \paragraph{Titled paragraphs.} You can use titled paragraphs if and 
% only if the title covers exactly one paragraph. Such paragraphs must be
% separated from the preceding content by at least 3pt, and no more than
% 6pt. The title must be in 10pt bold font and ended with a period. 
% After that, a 1em horizontal space must follow the title before 
% the paragraph's text.

% In \LaTeX{} titled paragraphs must be typeset using
% \begin{quote}
% {\tt \textbackslash{}paragraph\{Title.\} text} .
% \end{quote}

% \subsubsection{Acknowledgements}

% You may include an unnumbered acknowledgments section, including
% acknowledgments of help from colleagues, financial support, and
% permission to publish. If present, acknowledgements must be in a dedicated,
% unnumbered section appearing after all regular sections but before any
% appendices or references.

% Use 
% \begin{quote}
%     {\tt \textbackslash{}section*\{Acknowledgements\}})
% \end{quote}
% to typeset the acknowledgements section in \LaTeX{}.

% \subsubsection{Appendices}

% Any appendices directly follow the text and look like sections, except
% that they are numbered with capital letters instead of arabic
% numerals. See this document for an example.

% \subsubsection{References}

% The references section is headed ``References'', printed in the same
% style as a section heading but without a number. A sample list of
% references is given at the end of these instructions. Use a consistent
% format for references. The reference list should not include unpublished
% work.

% \subsection{Citations}

% Citations within the text should include the author's last name and
% the year of publication, for example~\cite{gottlob:nonmon}.  Append
% lowercase letters to the year in cases of ambiguity.  Treat multiple
% authors as in the following examples:~\cite{abelson-et-al:scheme}
% or~\cite{bgf:Lixto} (for more than two authors) and
% \cite{brachman-schmolze:kl-one} (for two authors).  If the author
% portion of a citation is obvious, omit it, e.g.,
% Nebel~\shortcite{nebel:jair-2000}.  Collapse multiple citations as
% follows:~\cite{gls:hypertrees,levesque:functional-foundations}.
% \nocite{abelson-et-al:scheme}
% \nocite{bgf:Lixto}
% \nocite{brachman-schmolze:kl-one}
% \nocite{gottlob:nonmon}
% \nocite{gls:hypertrees}
% \nocite{levesque:functional-foundations}
% \nocite{levesque:belief}
% \nocite{nebel:jair-2000}

% \subsection{Footnotes}

% Place footnotes at the bottom of the page in a 9-point font.  Refer to
% them with superscript numbers.\footnote{This is how your footnotes
% should appear.} Separate them from the text by a short
% line.\footnote{Note the line separating these footnotes from the
% text.} Avoid footnotes as much as possible; they interrupt the flow of
% the text.

% \section{Illustrations}

% Place all illustrations (figures, drawings, tables, and photographs)
% throughout the paper at the places where they are first discussed,
% rather than at the end of the paper.

% They should be floated to the top (preferred) or bottom of the page, 
% unless they are an integral part 
% of your narrative flow. When placed at the bottom or top of
% a page, illustrations may run across both columns, but not when they
% appear inline.

% Illustrations must be rendered electronically or scanned and placed
% directly in your document. All illustrations should be understandable when printed in black and
% white, albeit you can use colors to enhance them. Line weights should
% be 1/2-point or thicker. Avoid screens and superimposing type on
% patterns as these effects may not reproduce well.

% Number illustrations sequentially. Use references of the following
% form: Figure 1, Table 2, etc. Place illustration numbers and captions
% under illustrations. Leave a margin of 1/4-inch around the area
% covered by the illustration and caption.  Use 9-point type for
% captions, labels, and other text in illustrations. Captions should always appear below the illustration.

% \section{Tables}

% Tables are considered illustrations containing data. Therefore, they should also appear floated to the top (preferably) or bottom of the page, and with the captions below them.

% \begin{table}
% \centering
% \begin{tabular}{lll}
% \hline
% Scenario  & $\delta$ & Runtime \\
% \hline
% Paris       & 0.1s  & 13.65ms     \\
% Paris       & 0.2s  & 0.01ms      \\
% New York    & 0.1s  & 92.50ms     \\
% Singapore   & 0.1s  & 33.33ms     \\
% Singapore   & 0.2s  & 23.01ms     \\
% \hline
% \end{tabular}
% \caption{Latex default table}
% \label{tab:plain}
% \end{table}

% \begin{table}
% \centering
% \begin{tabular}{lrr}  
% \toprule
% Scenario  & $\delta$ (s) & Runtime (ms) \\
% \midrule
% Paris       & 0.1  & 13.65      \\
%             & 0.2  & 0.01       \\
% New York    & 0.1  & 92.50      \\
% Singapore   & 0.1  & 33.33      \\
%             & 0.2  & 23.01      \\
% \bottomrule
% \end{tabular}
% \caption{Booktabs table}
% \label{tab:booktabs}
% \end{table}

% If you are using \LaTeX, you should use the {\tt booktabs} package, because it produces better tables than the standard ones. Compare Tables \ref{tab:plain} and~\ref{tab:booktabs}. The latter is clearly more readable for three reasons:

% \begin{enumerate}
%     \item The styling is better thanks to using the {\tt booktabs} rulers instead of the default ones.
%     \item Numeric columns are right-aligned, making it easier to compare the numbers. Make sure to also right-align the corresponding headers, and to use the same precision for all numbers.
%     \item We avoid unnecessary repetition, both between lines (no need to repeat the scenario name in this case) as well as in the content (units can be shown in the column header).
% \end{enumerate}

% \section{Formulas}

% IJCAI's two-column format makes it difficult to typeset long formulas. A usual temptation is to reduce the size of the formula by using the {\tt small} or {\tt tiny} sizes. This doesn't work correctly with the current \LaTeX{} versions, breaking the line spacing of the preceding paragraphs and title, as well as the equation number sizes. The following equation demonstrates the effects (notice that this entire paragraph looks badly formatted):
% %
% \begin{tiny}
% \begin{equation}
%     x = \prod_{i=1}^n \sum_{j=1}^n j_i + \prod_{i=1}^n \sum_{j=1}^n i_j + \prod_{i=1}^n \sum_{j=1}^n j_i + \prod_{i=1}^n \sum_{j=1}^n i_j + \prod_{i=1}^n \sum_{j=1}^n j_i
% \end{equation}
% \end{tiny}%

% Reducing formula sizes this way is strictly forbidden. We {\bf strongly} recommend authors to split formulas in multiple lines when they don't fit in a single line. This is the easiest approach to typeset those formulas and provides the most readable output%
% %
% \begin{align}
%     x =& \prod_{i=1}^n \sum_{j=1}^n j_i + \prod_{i=1}^n \sum_{j=1}^n i_j + \prod_{i=1}^n \sum_{j=1}^n j_i + \prod_{i=1}^n \sum_{j=1}^n i_j + \nonumber\\
%     + & \prod_{i=1}^n \sum_{j=1}^n j_i
% \end{align}%

% If a line is just slightly longer than the column width, you may use the {\tt resizebox} environment on that equation. The result looks better and doesn't interfere with the paragraph's line spacing: %
% \begin{equation}
% \resizebox{.91\linewidth}{!}{$
%     \displaystyle
%     x = \prod_{i=1}^n \sum_{j=1}^n j_i + \prod_{i=1}^n \sum_{j=1}^n i_j + \prod_{i=1}^n \sum_{j=1}^n j_i + \prod_{i=1}^n \sum_{j=1}^n i_j + \prod_{i=1}^n \sum_{j=1}^n j_i
% $}
% \end{equation}%

% This last solution may have to be adapted if you use different equation environments, but it can generally be made to work. Please notice that in any case:

% \begin{itemize}
%     \item Equation numbers must be in the same font and size than the main text (10pt).
%     \item Your formula's main symbols should not be smaller than {\small small} text (9pt).
% \end{itemize}

% For instance, the formula
% %
% \begin{equation}
%     \resizebox{.91\linewidth}{!}{$
%     \displaystyle
%     x = \prod_{i=1}^n \sum_{j=1}^n j_i + \prod_{i=1}^n \sum_{j=1}^n i_j + \prod_{i=1}^n \sum_{j=1}^n j_i + \prod_{i=1}^n \sum_{j=1}^n i_j + \prod_{i=1}^n \sum_{j=1}^n j_i + \prod_{i=1}^n \sum_{j=1}^n i_j
% $}
% \end{equation}
% % 
% would not be acceptable because the text is too small.

% \section{Algorithms and Listings}

% Algorithms and listings are a special kind of figures. Like all illustrations, they should appear floated to the top (preferably) or bottom of the page. However, their caption should appear in the header, left-justified and enclosed between horizontal lines, as shown in Algorithm~\ref{alg:algorithm}. The algorithm body should be terminated with another horizontal line. It is up to the authors to decide whether to show line numbers or not, how to format comments, etc.

% In \LaTeX{} algorithms may be typeset using the {\tt algorithm} and {\tt algorithmic} packages, but you can also use one of the many other packages for the task.  

% \begin{algorithm}[tb]
% \caption{Example algorithm}
% \label{alg:algorithm}
% \textbf{Input}: Your algorithm's input\\
% \textbf{Parameter}: Optional list of parameters\\
% \textbf{Output}: Your algorithm's output
% \begin{algorithmic}[1] %[1] enables line numbers
% \STATE Let $t=0$.
% \WHILE{condition}
% \STATE Do some action.
% \IF {conditional}
% \STATE Perform task A.
% \ELSE
% \STATE Perform task B.
% \ENDIF
% \ENDWHILE
% \STATE \textbf{return} solution
% \end{algorithmic}
% \end{algorithm}

% \section*{Acknowledgments}

% The preparation of these instructions and the \LaTeX{} and Bib\TeX{}
% files that implement them was supported by Schlumberger Palo Alto
% Research, AT\&T Bell Laboratories, and Morgan Kaufmann Publishers.
% Preparation of the Microsoft Word file was supported by IJCAI.  An
% early version of this document was created by Shirley Jowell and Peter
% F. Patel-Schneider.  It was subsequently modified by Jennifer
% Ballentine and Thomas Dean, Bernhard Nebel, Daniel Pagenstecher,
% Kurt Steinkraus, Toby Walsh and Carles Sierra. The current version 
% has been prepared by Marc Pujol-Gonzalez and Francisco Cruz-Mencia.

% \appendix

% \section{\LaTeX{} and Word Style Files}\label{stylefiles}

% The \LaTeX{} and Word style files are available on the IJCAI--19
% website, \url{http://www.ijcai19.org}.
% These style files implement the formatting instructions in this
% document.

% The \LaTeX{} files are {\tt ijcai19.sty} and {\tt ijcai19.tex}, and
% the Bib\TeX{} files are {\tt named.bst} and {\tt ijcai19.bib}. The
% \LaTeX{} style file is for version 2e of \LaTeX{}, and the Bib\TeX{}
% style file is for version 0.99c of Bib\TeX{} ({\em not} version
% 0.98i). The {\tt ijcai19.sty} style differs from the {\tt
% ijcai18.sty} file used for IJCAI--18.

% The Microsoft Word style file consists of a single file, {\tt
% ijcai19.doc}. This template differs from the one used for
% IJCAI--18.

% These Microsoft Word and \LaTeX{} files contain the source of the
% present document and may serve as a formatting sample.  

% Further information on using these styles for the preparation of
% papers for IJCAI--19 can be obtained by contacting {\tt
% pcchair@ijcai19.org}.

%% The file named.bst is a bibliography style file for BibTeX 0.99c
\bibliographystyle{named}
\bibliography{ijcai19}

\end{document}

